% Options for packages loaded elsewhere
\PassOptionsToPackage{unicode}{hyperref}
\PassOptionsToPackage{hyphens}{url}
%
\documentclass[
]{book}
\usepackage{amsmath,amssymb}
\usepackage{lmodern}
\usepackage{iftex}
\ifPDFTeX
  \usepackage[T1]{fontenc}
  \usepackage[utf8]{inputenc}
  \usepackage{textcomp} % provide euro and other symbols
\else % if luatex or xetex
  \usepackage{unicode-math}
  \defaultfontfeatures{Scale=MatchLowercase}
  \defaultfontfeatures[\rmfamily]{Ligatures=TeX,Scale=1}
\fi
% Use upquote if available, for straight quotes in verbatim environments
\IfFileExists{upquote.sty}{\usepackage{upquote}}{}
\IfFileExists{microtype.sty}{% use microtype if available
  \usepackage[]{microtype}
  \UseMicrotypeSet[protrusion]{basicmath} % disable protrusion for tt fonts
}{}
\makeatletter
\@ifundefined{KOMAClassName}{% if non-KOMA class
  \IfFileExists{parskip.sty}{%
    \usepackage{parskip}
  }{% else
    \setlength{\parindent}{0pt}
    \setlength{\parskip}{6pt plus 2pt minus 1pt}}
}{% if KOMA class
  \KOMAoptions{parskip=half}}
\makeatother
\usepackage{xcolor}
\usepackage{longtable,booktabs,array}
\usepackage{calc} % for calculating minipage widths
% Correct order of tables after \paragraph or \subparagraph
\usepackage{etoolbox}
\makeatletter
\patchcmd\longtable{\par}{\if@noskipsec\mbox{}\fi\par}{}{}
\makeatother
% Allow footnotes in longtable head/foot
\IfFileExists{footnotehyper.sty}{\usepackage{footnotehyper}}{\usepackage{footnote}}
\makesavenoteenv{longtable}
\usepackage{graphicx}
\makeatletter
\def\maxwidth{\ifdim\Gin@nat@width>\linewidth\linewidth\else\Gin@nat@width\fi}
\def\maxheight{\ifdim\Gin@nat@height>\textheight\textheight\else\Gin@nat@height\fi}
\makeatother
% Scale images if necessary, so that they will not overflow the page
% margins by default, and it is still possible to overwrite the defaults
% using explicit options in \includegraphics[width, height, ...]{}
\setkeys{Gin}{width=\maxwidth,height=\maxheight,keepaspectratio}
% Set default figure placement to htbp
\makeatletter
\def\fps@figure{htbp}
\makeatother
\setlength{\emergencystretch}{3em} % prevent overfull lines
\providecommand{\tightlist}{%
  \setlength{\itemsep}{0pt}\setlength{\parskip}{0pt}}
\setcounter{secnumdepth}{5}
\usepackage{booktabs}
\ifLuaTeX
  \usepackage{selnolig}  % disable illegal ligatures
\fi
\usepackage[]{natbib}
\bibliographystyle{apalike}
\IfFileExists{bookmark.sty}{\usepackage{bookmark}}{\usepackage{hyperref}}
\IfFileExists{xurl.sty}{\usepackage{xurl}}{} % add URL line breaks if available
\urlstyle{same} % disable monospaced font for URLs
\hypersetup{
  pdftitle={BIOL 3295: Population and Evolutionary Ecology, Winter 2023},
  pdfauthor={Amy Hurford},
  hidelinks,
  pdfcreator={LaTeX via pandoc}}

\title{BIOL 3295: Population and Evolutionary Ecology, Winter 2023}
\author{Amy Hurford}
\date{2023-01-05}

\begin{document}
\maketitle

{
\setcounter{tocdepth}{1}
\tableofcontents
}
\hypertarget{syllabus}{%
\chapter{Syllabus}\label{syllabus}}

\hypertarget{instructor-information}{%
\section{Instructor Information}\label{instructor-information}}

Instructor: Dr.~Amy Hurford\\
Office: CSF 4338\\
Email: \href{mailto:ahurford@mun.ca}{\nolinkurl{ahurford@mun.ca}}\\
I will try to reply to emails within 24 hours (excluding evenings, weekends and holidays).
Office hours: Tuesday 1-2pm; Thursday 1-2pm

\hypertarget{course-information}{%
\section{Course Information}\label{course-information}}

TR 12.00-12.50pm\\
F 1-1.50pm\\
Classroom: SN3060

All Course Announcements will be made on BrightSpace. Should lectures be remote a WebEx link will be provided on BrightSpace.

Course description:\\
Population and Evolutionary Ecology is an introduction to the theory and principles of evolutionary ecology and population dynamics. Pre-requisites: BIOL 2600; at least one of BIOL 2010, 2122 or 2210.\\

Course format:\\
The course consists of lectures, 4 data analysis assignments, 2 exams and a final exam.~

Course expectations:\\
Please attend lectures and respect the learning environment of other students.\\

Learning goals:\\
The course content emphasizes a deeper understanding of fewer concepts. You have seen much of the course material in pre-requisite courses. In this course, I will revisit the models, clarify the assumptions and when they are appropriate, and we will fit the models to data to estimate parameters. By the end of the course, I hope that if you were given population data, that you would know the key quantities that you might estimate, and could do the analysis.

Required Text and Resources:\\
The course materials are online at \url{https://ahurford.github.io/BIOL-3295-Winter-2023/}.

If you wish to use your own computer for assignments you should install \texttt{R} and \texttt{RStudio} (see also \href{https://ahurford.github.io/quant-guide-all-courses/install.html}{here}).

\hypertarget{method-of-evaluation}{%
\section{Method of Evaluation}\label{method-of-evaluation}}

\begin{itemize}
\tightlist
\item
  4 Assignments - 20\%
\item
  2 Exams - 40\%
\item
  Final Exam - 40\%
\end{itemize}

Late assignments, labs, and missed midterms, and final exams will be accommodated as described by University Regulation 6.7.3 and 6.7.5 (see \url{https://www.mun.ca/regoff/calendar/sectionNo=REGS-0474} for Regulations).

\hypertarget{additional-policies}{%
\section{Additional Policies}\label{additional-policies}}

\hypertarget{accommodation-of-students-with-disabilities}{%
\subsection{Accommodation of students with disabilities}\label{accommodation-of-students-with-disabilities}}

Memorial University of Newfoundland is committed to supporting inclusive education based on the principles of equity, accessibility and collaboration. Accommodations are provided within the scope of the University Policies for the Accommodations for Students with Disabilities see \url{www.mun.ca/policy/site/policy.php?id=239}. Students who may need an academic accommodation are asked to initiate the request with the Glenn Roy Blundon Centre at the earliest opportunity (see \url{www.mun.ca/blundon} for more information).

\hypertarget{academic-misconduct}{%
\subsection{Academic misconduct}\label{academic-misconduct}}

Students are expected to adhere to those principles, which constitute proper academic conduct. A student has the responsibility to know which actions, as described under Academic Offences in the University Regulations, could be construed as dishonest or improper. Students found guilty of an academic offence may be subject to a number of penalties commensurate with the offence including reprimand, reduction of grade, probation, suspension or expulsion from the University. For more information regarding this policy, students should refer to University Regulation 6.12.

\hypertarget{equity-and-diversity}{%
\subsection{Equity and Diversity}\label{equity-and-diversity}}

A safe learning environment will be provided for all students regardless of race, colour, nationality, ethnic origin, social origin, religious creed, religion, age, disability, disfigurement, sex (including pregnancy), sexual orientation, gender identity, gender expression, marital status, family status, source of income or political opinion.

You should not photograph or record myself, teaching assistants, or other students in the class without first obtaining permission. Accommodation will be made for students with special needs.

The sound should be turned off on phones and computers during class.

\hypertarget{additional-supports}{%
\section{Additional Supports}\label{additional-supports}}

Resources for additional support can be found at:

\begin{itemize}
\item
  \url{www.mun.ca/currentstudents/student/}
\item
  \url{https://munsu.ca/resource-centres/}
\end{itemize}

\hypertarget{schedule}{%
\chapter{Schedule}\label{schedule}}

\begin{itemize}
\tightlist
\item
  Thurs Jan 5: Introduction
\item
  Fri Jan 6: Population biology with discrete and continuous variables
\item
  Tues Jan 10: Introduction to Rmarkdown and tidyverse \textbf{Assignment 1 is assigned}
\item
  Thurs Jan 12: Geometric growth
\item
  Fri Jan 13: Geometric growth
\item
  Tues Jan 17: Numerical solutions and graphing population data \textbf{Assignment 1 is due \& Assignment 2 is assigned}
\item
  Thurs Jan 19: Exponential growth
\item
  Fri Jan 20: Exponential growth
\item
  Tues Jan 24: Density dependence and logistic growth
  \textbf{Assignement 2 is due}
\item
  Thurs Jan 26: Density dependence and logistic growth
\item
  Fri Jan 27: Density dependence and logistic growth
\item
  Tues Jan 31: Discrete time density dependence
\item
  Thurs Feb 2: \textbf{EXAM I}
\item
  Fri Feb 3: Age-structured models
\item
  Tues Feb 7: Stage-structured models
\item
  Thurs Feb 9: Stage-structured models
\item
  Fri Feb 10: Stage-structured models
\item
  Tues Feb 14: Numerical analysis of stage-structured models \textbf{Assignment 3 is assigned}
\item
  Thurs Feb 16: Density dependence in stage-structured models
\item
  Fri Feb 17: Metapopulation models
  WINTER BREAK
\item
  Tues Feb 28: Continuous space models \textbf{Assignment 3 is due}
\item
  Thurs Mar 2: Spatially explicit models in population biology
\item
  Fri Mar 3: Population dynamics in a warming world
\item
  Tues Mar 7: Spatially explicit population dynamics in a warming world
\item
  Thurs Mar 9: Disease dynamics
\item
  Fri Mar 10: The net reproduction number
\item
  Tues Mar 14: Overview of models in population biology
\item
  Thurs Mar 16: \textbf{EXAM II}
\item
  Fri Mar 17: What is evolutionary ecology?
\item
  Tues Mar 21: Haploid selection model
\item
  Thur Mar 23: \href{https://www.zoology.ubc.ca/~otto/Talks/SSE2022_Otto.pdf}{Selection coefficients for COVID-19 variants}
\item
  Fri Mar 24: Estimating selection coefficients \textbf{Assignment 4 is assigned}
\item
  Tues Mar 28: The evolutionary ecology of pathogens
\item
  Thurs Mar 30: The evolutionary ecology of COVID-19
\item
  Fri Mar 31: The evolutionary ecology of hosts \textbf{Assignment 4 is due}
\item
  Tues Apr 3: The evolution of reproductive effort in plants
\item
  Thurs Apr 5: Evolutionarily stable and convergent stable strategies
\item
  Fri Apr 6: Review
\end{itemize}

\end{document}
