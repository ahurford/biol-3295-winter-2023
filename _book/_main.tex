% Options for packages loaded elsewhere
\PassOptionsToPackage{unicode}{hyperref}
\PassOptionsToPackage{hyphens}{url}
%
\documentclass[
]{book}
\usepackage{amsmath,amssymb}
\usepackage{lmodern}
\usepackage{iftex}
\ifPDFTeX
  \usepackage[T1]{fontenc}
  \usepackage[utf8]{inputenc}
  \usepackage{textcomp} % provide euro and other symbols
\else % if luatex or xetex
  \usepackage{unicode-math}
  \defaultfontfeatures{Scale=MatchLowercase}
  \defaultfontfeatures[\rmfamily]{Ligatures=TeX,Scale=1}
\fi
% Use upquote if available, for straight quotes in verbatim environments
\IfFileExists{upquote.sty}{\usepackage{upquote}}{}
\IfFileExists{microtype.sty}{% use microtype if available
  \usepackage[]{microtype}
  \UseMicrotypeSet[protrusion]{basicmath} % disable protrusion for tt fonts
}{}
\makeatletter
\@ifundefined{KOMAClassName}{% if non-KOMA class
  \IfFileExists{parskip.sty}{%
    \usepackage{parskip}
  }{% else
    \setlength{\parindent}{0pt}
    \setlength{\parskip}{6pt plus 2pt minus 1pt}}
}{% if KOMA class
  \KOMAoptions{parskip=half}}
\makeatother
\usepackage{xcolor}
\usepackage{color}
\usepackage{fancyvrb}
\newcommand{\VerbBar}{|}
\newcommand{\VERB}{\Verb[commandchars=\\\{\}]}
\DefineVerbatimEnvironment{Highlighting}{Verbatim}{commandchars=\\\{\}}
% Add ',fontsize=\small' for more characters per line
\usepackage{framed}
\definecolor{shadecolor}{RGB}{248,248,248}
\newenvironment{Shaded}{\begin{snugshade}}{\end{snugshade}}
\newcommand{\AlertTok}[1]{\textcolor[rgb]{0.94,0.16,0.16}{#1}}
\newcommand{\AnnotationTok}[1]{\textcolor[rgb]{0.56,0.35,0.01}{\textbf{\textit{#1}}}}
\newcommand{\AttributeTok}[1]{\textcolor[rgb]{0.77,0.63,0.00}{#1}}
\newcommand{\BaseNTok}[1]{\textcolor[rgb]{0.00,0.00,0.81}{#1}}
\newcommand{\BuiltInTok}[1]{#1}
\newcommand{\CharTok}[1]{\textcolor[rgb]{0.31,0.60,0.02}{#1}}
\newcommand{\CommentTok}[1]{\textcolor[rgb]{0.56,0.35,0.01}{\textit{#1}}}
\newcommand{\CommentVarTok}[1]{\textcolor[rgb]{0.56,0.35,0.01}{\textbf{\textit{#1}}}}
\newcommand{\ConstantTok}[1]{\textcolor[rgb]{0.00,0.00,0.00}{#1}}
\newcommand{\ControlFlowTok}[1]{\textcolor[rgb]{0.13,0.29,0.53}{\textbf{#1}}}
\newcommand{\DataTypeTok}[1]{\textcolor[rgb]{0.13,0.29,0.53}{#1}}
\newcommand{\DecValTok}[1]{\textcolor[rgb]{0.00,0.00,0.81}{#1}}
\newcommand{\DocumentationTok}[1]{\textcolor[rgb]{0.56,0.35,0.01}{\textbf{\textit{#1}}}}
\newcommand{\ErrorTok}[1]{\textcolor[rgb]{0.64,0.00,0.00}{\textbf{#1}}}
\newcommand{\ExtensionTok}[1]{#1}
\newcommand{\FloatTok}[1]{\textcolor[rgb]{0.00,0.00,0.81}{#1}}
\newcommand{\FunctionTok}[1]{\textcolor[rgb]{0.00,0.00,0.00}{#1}}
\newcommand{\ImportTok}[1]{#1}
\newcommand{\InformationTok}[1]{\textcolor[rgb]{0.56,0.35,0.01}{\textbf{\textit{#1}}}}
\newcommand{\KeywordTok}[1]{\textcolor[rgb]{0.13,0.29,0.53}{\textbf{#1}}}
\newcommand{\NormalTok}[1]{#1}
\newcommand{\OperatorTok}[1]{\textcolor[rgb]{0.81,0.36,0.00}{\textbf{#1}}}
\newcommand{\OtherTok}[1]{\textcolor[rgb]{0.56,0.35,0.01}{#1}}
\newcommand{\PreprocessorTok}[1]{\textcolor[rgb]{0.56,0.35,0.01}{\textit{#1}}}
\newcommand{\RegionMarkerTok}[1]{#1}
\newcommand{\SpecialCharTok}[1]{\textcolor[rgb]{0.00,0.00,0.00}{#1}}
\newcommand{\SpecialStringTok}[1]{\textcolor[rgb]{0.31,0.60,0.02}{#1}}
\newcommand{\StringTok}[1]{\textcolor[rgb]{0.31,0.60,0.02}{#1}}
\newcommand{\VariableTok}[1]{\textcolor[rgb]{0.00,0.00,0.00}{#1}}
\newcommand{\VerbatimStringTok}[1]{\textcolor[rgb]{0.31,0.60,0.02}{#1}}
\newcommand{\WarningTok}[1]{\textcolor[rgb]{0.56,0.35,0.01}{\textbf{\textit{#1}}}}
\usepackage{longtable,booktabs,array}
\usepackage{calc} % for calculating minipage widths
% Correct order of tables after \paragraph or \subparagraph
\usepackage{etoolbox}
\makeatletter
\patchcmd\longtable{\par}{\if@noskipsec\mbox{}\fi\par}{}{}
\makeatother
% Allow footnotes in longtable head/foot
\IfFileExists{footnotehyper.sty}{\usepackage{footnotehyper}}{\usepackage{footnote}}
\makesavenoteenv{longtable}
\usepackage{graphicx}
\makeatletter
\def\maxwidth{\ifdim\Gin@nat@width>\linewidth\linewidth\else\Gin@nat@width\fi}
\def\maxheight{\ifdim\Gin@nat@height>\textheight\textheight\else\Gin@nat@height\fi}
\makeatother
% Scale images if necessary, so that they will not overflow the page
% margins by default, and it is still possible to overwrite the defaults
% using explicit options in \includegraphics[width, height, ...]{}
\setkeys{Gin}{width=\maxwidth,height=\maxheight,keepaspectratio}
% Set default figure placement to htbp
\makeatletter
\def\fps@figure{htbp}
\makeatother
\setlength{\emergencystretch}{3em} % prevent overfull lines
\providecommand{\tightlist}{%
  \setlength{\itemsep}{0pt}\setlength{\parskip}{0pt}}
\setcounter{secnumdepth}{5}
\usepackage{booktabs}
\ifLuaTeX
  \usepackage{selnolig}  % disable illegal ligatures
\fi
\usepackage[]{natbib}
\bibliographystyle{plainnat}
\IfFileExists{bookmark.sty}{\usepackage{bookmark}}{\usepackage{hyperref}}
\IfFileExists{xurl.sty}{\usepackage{xurl}}{} % add URL line breaks if available
\urlstyle{same} % disable monospaced font for URLs
\hypersetup{
  pdftitle={BIOL 3295, Winter 2023},
  pdfauthor={Amy Hurford},
  hidelinks,
  pdfcreator={LaTeX via pandoc}}

\title{BIOL 3295, Winter 2023}
\author{Amy Hurford}
\date{2023-01-04}

\usepackage{amsthm}
\newtheorem{theorem}{Theorem}[chapter]
\newtheorem{lemma}{Lemma}[chapter]
\newtheorem{corollary}{Corollary}[chapter]
\newtheorem{proposition}{Proposition}[chapter]
\newtheorem{conjecture}{Conjecture}[chapter]
\theoremstyle{definition}
\newtheorem{definition}{Definition}[chapter]
\theoremstyle{definition}
\newtheorem{example}{Example}[chapter]
\theoremstyle{definition}
\newtheorem{exercise}{Exercise}[chapter]
\theoremstyle{definition}
\newtheorem{hypothesis}{Hypothesis}[chapter]
\theoremstyle{remark}
\newtheorem*{remark}{Remark}
\newtheorem*{solution}{Solution}
\begin{document}
\maketitle

{
\setcounter{tocdepth}{1}
\tableofcontents
}
\hypertarget{about}{%
\chapter{About}\label{about}}

This is a \emph{sample} book written in \textbf{Markdown}. You can use anything that Pandoc's Markdown supports; for example, a math equation \(a^2 + b^2 = c^2\).

\hypertarget{usage}{%
\section{Usage}\label{usage}}

Each \textbf{bookdown} chapter is an .Rmd file, and each .Rmd file can contain one (and only one) chapter. A chapter \emph{must} start with a first-level heading: \texttt{\#\ A\ good\ chapter}, and can contain one (and only one) first-level heading.

Use second-level and higher headings within chapters like: \texttt{\#\#\ A\ short\ section} or \texttt{\#\#\#\ An\ even\ shorter\ section}.

The \texttt{index.Rmd} file is required, and is also your first book chapter. It will be the homepage when you render the book.

\hypertarget{render-book}{%
\section{Render book}\label{render-book}}

You can render the HTML version of this example book without changing anything:

\begin{enumerate}
\def\labelenumi{\arabic{enumi}.}
\item
  Find the \textbf{Build} pane in the RStudio IDE, and
\item
  Click on \textbf{Build Book}, then select your output format, or select ``All formats'' if you'd like to use multiple formats from the same book source files.
\end{enumerate}

Or build the book from the R console:

\begin{Shaded}
\begin{Highlighting}[]
\NormalTok{bookdown}\SpecialCharTok{::}\FunctionTok{render\_book}\NormalTok{()}
\end{Highlighting}
\end{Shaded}

To render this example to PDF as a \texttt{bookdown::pdf\_book}, you'll need to install XeLaTeX. You are recommended to install TinyTeX (which includes XeLaTeX): \url{https://yihui.org/tinytex/}.

\hypertarget{preview-book}{%
\section{Preview book}\label{preview-book}}

As you work, you may start a local server to live preview this HTML book. This preview will update as you edit the book when you save individual .Rmd files. You can start the server in a work session by using the RStudio add-in ``Preview book'', or from the R console:

\begin{Shaded}
\begin{Highlighting}[]
\NormalTok{bookdown}\SpecialCharTok{::}\FunctionTok{serve\_book}\NormalTok{()}
\end{Highlighting}
\end{Shaded}

\hypertarget{schedule}{%
\chapter{Schedule}\label{schedule}}

\begin{itemize}
\tightlist
\item
  Thurs Jan 5: Introduction
\item
  Fri Jan 6: Population biology with discrete and continuous variables
\item
  Tues Jan 10: Introduction to Rmarkdown and tidyverse \textbf{Assignment 1 is assigned}
\item
  Thurs Jan 12: Geometric growth
\item
  Fri Jan 13: Geometric growth
\item
  Tues Jan 17: Numerical solutions and graphing population data \textbf{Assignment 1 is due \& Assignment 2 is assigned}
\item
  Thurs Jan 19: Exponential growth
\item
  Fri Jan 20: Exponential growth
\item
  Tues Jan 24: Density dependence and logistic growth
  \textbf{Assignement 2 is due}
\item
  Thurs Jan 26: Density dependence and logistic growth
\item
  Fri Jan 27: Density dependence and logistic growth
\item
  Tues Jan 31: Discrete time density dependence
\item
  Thurs Feb 2: \textbf{EXAM I}
\item
  Fri Feb 3: Age-structured models
\item
  Tues Feb 7: Stage-structured models
\item
  Thurs Feb 9: Stage-structured models
\item
  Fri Feb 10: Stage-structured models
\item
  Tues Feb 14: Numerical analysis of stage-structured models \textbf{Assignment 3 is assigned}
\item
  Thurs Feb 16: Density dependence in stage-structured models
\item
  Fri Feb 17: Metapopulation models
  WINTER BREAK
\item
  Tues Feb 28: Continuous space models \textbf{Assignment 3 is due}
\item
  Thurs Mar 2: Spatially explicit models in population biology
\item
  Fri Mar 3: Population dynamics in a warming world
\item
  Tues Mar 7: Spatially explicit population dynamics in a warming world
\item
  Thurs Mar 9: Disease dynamics
\item
  Fri Mar 10: The net reproduction number
\item
  Tues Mar 14: Overview of models in population biology
\item
  Thurs Mar 16: \textbf{EXAM II}
\item
  Fri Mar 17: What is evolutionary ecology?
\item
  Tues Mar 21: Haploid selection model
\item
  Thur Mar 23: \href{https://www.zoology.ubc.ca/~otto/Talks/SSE2022_Otto.pdf}{Selection coefficients for COVID-19 variants}
\item
  Fri Mar 24: Estimating selection coefficients \textbf{Assignment 4 is assigned}
\item
  Tues Mar 28: The evolutionary ecology of pathogens
\item
  Thurs Mar 30: The evolutionary ecology of COVID-19
\item
  Fri Mar 31: The evolutionary ecology of hosts \textbf{Assignment 4 is due}
\item
  Tues Apr 3: The evolution of reproductive effort in plants
\item
  Thurs Apr 5: Evolutionarily stable and convergent stable strategies
\item
  Fri Apr 6: Review
\end{itemize}

\hypertarget{blocks}{%
\chapter{Blocks}\label{blocks}}

\hypertarget{equations}{%
\section{Equations}\label{equations}}

Here is an equation.

\begin{equation} 
  f\left(k\right) = \binom{n}{k} p^k\left(1-p\right)^{n-k}
  \label{eq:binom}
\end{equation}

You may refer to using \texttt{\textbackslash{}@ref(eq:binom)}, like see Equation \eqref{eq:binom}.

\hypertarget{theorems-and-proofs}{%
\section{Theorems and proofs}\label{theorems-and-proofs}}

Labeled theorems can be referenced in text using \texttt{\textbackslash{}@ref(thm:tri)}, for example, check out this smart theorem \ref{thm:tri}.

\begin{theorem}
\protect\hypertarget{thm:tri}{}\label{thm:tri}For a right triangle, if \(c\) denotes the \emph{length} of the hypotenuse
and \(a\) and \(b\) denote the lengths of the \textbf{other} two sides, we have
\[a^2 + b^2 = c^2\]
\end{theorem}

Read more here \url{https://bookdown.org/yihui/bookdown/markdown-extensions-by-bookdown.html}.

\hypertarget{callout-blocks}{%
\section{Callout blocks}\label{callout-blocks}}

The R Markdown Cookbook provides more help on how to use custom blocks to design your own callouts: \url{https://bookdown.org/yihui/rmarkdown-cookbook/custom-blocks.html}

\hypertarget{footnotes-and-citations}{%
\chapter{Footnotes and citations}\label{footnotes-and-citations}}

\hypertarget{footnotes}{%
\section{Footnotes}\label{footnotes}}

Footnotes are put inside the square brackets after a caret \texttt{\^{}{[}{]}}. Like this one \footnote{This is a footnote.}.

\hypertarget{citations}{%
\section{Citations}\label{citations}}

Reference items in your bibliography file(s) using \texttt{@key}.

For example, we are using the \textbf{bookdown} package \citep{R-bookdown} (check out the last code chunk in index.Rmd to see how this citation key was added) in this sample book, which was built on top of R Markdown and \textbf{knitr} \citep{xie2015} (this citation was added manually in an external file book.bib).
Note that the \texttt{.bib} files need to be listed in the index.Rmd with the YAML \texttt{bibliography} key.

The RStudio Visual Markdown Editor can also make it easier to insert citations: \url{https://rstudio.github.io/visual-markdown-editing/\#/citations}

\hypertarget{parts}{%
\chapter{Parts}\label{parts}}

You can add parts to organize one or more book chapters together. Parts can be inserted at the top of an .Rmd file, before the first-level chapter heading in that same file.

Add a numbered part: \texttt{\#\ (PART)\ Act\ one\ \{-\}} (followed by \texttt{\#\ A\ chapter})

Add an unnumbered part: \texttt{\#\ (PART\textbackslash{}*)\ Act\ one\ \{-\}} (followed by \texttt{\#\ A\ chapter})

Add an appendix as a special kind of un-numbered part: \texttt{\#\ (APPENDIX)\ Other\ stuff\ \{-\}} (followed by \texttt{\#\ A\ chapter}). Chapters in an appendix are prepended with letters instead of numbers.

\hypertarget{sharing-your-book}{%
\chapter{Sharing your book}\label{sharing-your-book}}

\hypertarget{publishing}{%
\section{Publishing}\label{publishing}}

HTML books can be published online, see: \url{https://bookdown.org/yihui/bookdown/publishing.html}

\hypertarget{pages}{%
\section{404 pages}\label{pages}}

By default, users will be directed to a 404 page if they try to access a webpage that cannot be found. If you'd like to customize your 404 page instead of using the default, you may add either a \texttt{\_404.Rmd} or \texttt{\_404.md} file to your project root and use code and/or Markdown syntax.

\hypertarget{metadata-for-sharing}{%
\section{Metadata for sharing}\label{metadata-for-sharing}}

Bookdown HTML books will provide HTML metadata for social sharing on platforms like Twitter, Facebook, and LinkedIn, using information you provide in the \texttt{index.Rmd} YAML. To setup, set the \texttt{url} for your book and the path to your \texttt{cover-image} file. Your book's \texttt{title} and \texttt{description} are also used.

This \texttt{gitbook} uses the same social sharing data across all chapters in your book- all links shared will look the same.

Specify your book's source repository on GitHub using the \texttt{edit} key under the configuration options in the \texttt{\_output.yml} file, which allows users to suggest an edit by linking to a chapter's source file.

Read more about the features of this output format here:

\url{https://pkgs.rstudio.com/bookdown/reference/gitbook.html}

Or use:

\begin{Shaded}
\begin{Highlighting}[]
\NormalTok{?bookdown}\SpecialCharTok{::}\NormalTok{gitbook}
\end{Highlighting}
\end{Shaded}


  \bibliography{book.bib,packages.bib}

\end{document}
